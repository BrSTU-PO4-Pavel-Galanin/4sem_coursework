\section{Заключение}

Нейронные сети, технологии прошлого века, меняют работу целых отраслей. Одни возмодности нейронной сети поражают, а другие заставляют задуматься в их пользу как специалистов.

В результате курсового проекта была разработана нейронная сеть,
которая распознает рукописные цифры базы данных MNIST,
написанная на tensorflow,
которая использует надстройку Keras.

Архитектура сети представлена на листинге.

\begin{lstlisting}[language=Python]
    model = tensorflow.keras.Sequential([
        tensorflow.keras.layers.Flatten(input_shape=(28, 28, 1)),
        tensorflow.keras.layers.Dense(128, activation='relu'),
        tensorflow.keras.layers.Dense(10, activation='softmax')
    ])

    model.compile(
        optimizer=tensorflow.keras.optimizers.Adam(0.001),
        loss='categorical_crossentropy',
        metrics=['accuracy']
    )
\end{lstlisting}