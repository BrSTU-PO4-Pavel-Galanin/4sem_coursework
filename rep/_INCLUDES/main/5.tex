\section{Тестирование, сравнительный анализ и результаты исследования}

\subsection{Подбор шага для оптимизатора Adam}

\underline{Ожидаемый результат}:
подборка шага для наилутшей точности распознавания рукописной цифры.

\underline{Описнание}:

При компилировании модели можно выбрать оптимизатор. В ходе курсовой работы выбран оптимизатор Adam.

\begin{lstlisting}[language=Python]
    model.compile(
        optimizer='adam',
        ...
    )
\end{lstlisting}

Оптимизатор Adam по умолчанию имеет шаг $0.001$. Этот шаг можно поменять на свой, чем и будем заниматься, при проведении тестов.

\begin{lstlisting}[language=Python]
    model.compile(
        optimizer=tensorflow.keras.optimizers.Adam(0.001),
        ...
    )
\end{lstlisting}

\underline{Машина тестирования}:

\begin{table}[htbp]
    \centering
    \caption{Конфигурация машины}
    \label{tab:5_adam_step_machine}
    \begin{tabular}{|l|l|}
        \hline
        OS                      & Windows 10 Home                                   \\  \hline
        Processor               & AMD Ryzen 3 3250U with Radeon Graphics 2.60 GHz   \\  \hline
        Installed memory (RAM)  & 4 GB                                              \\  \hline
        System type             & 64-bit OS, x64-based processor                    \\  \hline
    \end{tabular}
\end{table}

\underline{Полученный результат}:

\begin{table}[!htbp]\footnotesize
    \centering
    \caption{Точность распознавания в зависимости от шага для алгоритма Adam}
    \label{tab:5_adam_step}
    \begin{tabular}{|l|*{10}{|l}||l|}
        \hline
        Шаг	&	\multicolumn{10}{|l|}{Точность при эксперименте №}	&	Сред.	\\	\hhline{|~|*{10}{-}|~|}
        adam	&	1	&	2	&	3	&	4	&	5	&	6	&	7	&	8	&	9	&	10	&	знач.	\\	
        \hline\hline
        default	&	0.9755	&	0.971	&	0.9753	&	0.9725	&	0.9726	&	0.9717	&	0.9737	&	0.9742	&	0.9732	&	0.976	&	0.9736	\\	\hline
        0.1	&	0.4418	&	0.6628	&	0.6077	&	0.5419	&	0.5823	&	0.5548	&	0.5499	&	0.5149	&	0.5213	&	0.6136	&	0.5591	\\	\hline
        0.01	&	0.9536	&	0.9593	&	0.9531	&	0.9571	&	0.9639	&	0.9519	&	0.9561	&	0.9613	&	0.9619	&	0.9538	&	0.9572	\\	\hline
        0.005	&	0.9673	&	0.9689	&	0.9685	&	0.9671	&	0.9681	&	0.9729	&	0.9626	&	0.9646	&	0.9648	&	0.9678	&	0.9673	\\	\hline
        0.001	&	0.9733	&	0.9732	&	0.9757	&	0.9741	&	0.9731	&	0.9714	&	0.973	&	0.972	&	0.9735	&	0.9732	&	0.9733	\\	\hline
        0.0005	&	0.9682	&	0.9693	&	0.9712	&	0.9701	&	0.9702	&	0.971	&	0.9706	&	0.9724	&	0.9693	&	0.9682	&	0.9701	\\	\hline
        0.0001	&	0.9487	&	0.9486	&	0.9492	&	0.9489	&	0.9487	&	0.9488	&	0.9499	&	0.9463	&	0.9484	&	0.9488	&	0.9486	\\	\hline
        1E-05	&	0.8905	&	0.8938	&	0.8929	&	0.8968	&	0.8905	&	0.8904	&	0.8923	&	0.8934	&	0.8932	&	0.8942	&	0.8928	\\	\hline
        1E-06	&	0.5722	&	0.5579	&	0.6092	&	0.5609	&	0.5961	&	0.5732	&	0.6217	&	0.5152	&	0.6123	&	0.5733	&	0.5792	\\	\hline
    \end{tabular}
\end{table}

Лутшая точность при шаге $0.001$. Этот шаг также является default шагом, что и видно по
таблице~\textbf{\ref{tab:5_adam_step} (стр. \pageref{tab:5_adam_step})},
так как $0.9736 \approx 0.9733$.